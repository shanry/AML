\documentclass[a4paper,UTF8]{article}
\usepackage{ctex}
\usepackage[margin=1.25in]{geometry}
\usepackage{color}
\usepackage{graphicx}
\usepackage{amssymb}
\usepackage{amsmath}
\usepackage{amsthm}
%\usepackage[thmmarks, amsmath, thref]{ntheorem}
\theoremstyle{definition}
\newtheorem*{solution}{Solution}
\newtheorem*{prove}{Proof}
\usepackage{multirow}
\usepackage{url}
\usepackage{enumerate}
\usepackage{algorithm}
\usepackage{algorithmic}
\renewcommand{\algorithmicrequire}{\textbf{Input:}}
\renewcommand{\algorithmicensure}{\textbf{Procedure:}}
\renewcommand\refname{参考文献}

%--

%--
\begin{document}
\title{实验2. 隐马尔科夫模型实践}
\author{MG1733099,周天烁,\url{tianshuo.zhou@smail.nju.edu.cn}}
\maketitle

\section*{综述}
	隐马尔可夫模型(Hidden Markov Model,HMM)是统计模型,它用来描述一个含有隐含未知参数的马尔可夫过程。其难点是从可观察的参数中确定该过程的隐含参数。然后利用这些参数来作进一步的分析,例如语音识别,词性标注,时间序列分析等。一个基本的隐马尔可夫模型主要以下五个部分组成,包括2个状态集合和3个概率矩阵:
    \begin{itemize}
    \item[1]  隐含状态 S
    \item[2]  可观测状态 O
    \item[3]  隐含状态转移概率矩阵 A
    \item[4]  观测状态转移概率矩阵 B
    \item[5]  初始状态概率矩阵 π
    \end{itemize}
    \paragraph{} 通过指定状态空间S、观测空间O和一组概率矩阵,就能确定一个隐马尔科夫模型,通常用$\lambda$=[A,B,π]来指代。给定一组观测序列,Baum Welch algorithm算法利用 EM algorithm和最大似然估计( maximum likelihood )来学习得到一组马尔科夫模型参数$\lambda$。该算法主要由Forward procedure、Backward procedure、Update三部分组成,实验二三分别实现其中的Forward procedure和Backward procedure,实验一则实现经典的维特比算法(Viterbi algorithm)来利用学习的参数进行股票的预测。
	
\section*{实验一.维特比算法}
	维特比算法采用动态规划思想,给定隐马尔可夫模型$\lambda$和观测序列O,该算法估计出最大可能的隐藏序列X。具体实现参考维基百科页面伪代码。

\section*{实验二.实现Forward Algorithm}
	Forward Algorithm是Baum Welch algorithm中的重要一步,用来估计给定某时刻的隐状态下最有可能的该时刻之前的观测序列。具体实现参考维基百科页面伪代码。

\section*{实验三.实现Backward Algorithm }
	Backward Algorithm是Baum Welch algorithm中的重要一步,和Forward Algorithm相反,该算法用来估计隐状态给的下的某时刻之后的最有可能的观测序列。具体实现参考维基百科页面伪代码。

\section*{附加任务  Do something EXTRA }
    利用以上实现的隐马尔科夫模型来进行自然语言处理的词性标注,具体实现参见POStag.py及其说明文件。

\end{document} 